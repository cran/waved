\HeaderA{maxithresh}{Returns maxiset threshold}{maxithresh}
\keyword{internal}{maxithresh}
\begin{Description}\relax
Returns maxiset threshold for specified resolution levels.
\end{Description}
\begin{Usage}
\begin{verbatim}
maxithresh(data, g, L = 2, F = (log2(length(data)) - 1), eta = sqrt(6))
\end{verbatim}
\end{Usage}
\begin{Arguments}
\begin{ldescription}
\item[\code{data}] ~~Describe \code{data} here~~ 
\item[\code{g}] ~~Describe \code{g} here~~ 
\item[\code{L}] ~~Describe \code{L} here~~ 
\item[\code{F}] ~~Describe \code{F} here~~ 
\item[\code{eta}] ~~Describe \code{eta} here~~ 
\end{ldescription}
\end{Arguments}
\begin{Details}\relax
~~ If necessary, more details than the description above ~~
\end{Details}
\begin{Value}
Returns maxiset threshold for coarse resolution level equal to L, 
and wavelet resolution levels L,...,F.
\end{Value}
\begin{Author}\relax
Marc Raimondo
\end{Author}
\begin{References}\relax
Johnstone, I., Kerkyacharian, G., Picard, D. and Raimondo, M.  (2004), 
`Wavelet deconvolution in a periodic
setting', {\em Journal of the Royal Statistical Society, Series B} {\bf
66}(3),~547--573.  with discussion pp.627-652.

Raimondo, M. and Stewart, M. (2006),
`The WaveD Transform in R', preprint, School and Mathematics and Statistics,
University of Sydney.
\end{References}
\begin{SeeAlso}\relax
\code{\LinkA{WaveD}{WaveD}}
\end{SeeAlso}
\begin{Examples}
\begin{ExampleCode}
library(waved)
data=waved.example(TRUE,FALSE)
maxithresh(data$lidar.noisy,data$g,L=3,F=7)
\end{ExampleCode}
\end{Examples}

