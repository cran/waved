\HeaderA{IFWT\_TI}{Inverse Forward Wavelet Transform (translation invariant).}{IFWT.Rul.TI}
\keyword{internal}{IFWT\_TI}
\begin{Description}\relax
Compute the Inverse Forward Wavelet Transform of a signal $f$ for the Meyer wavelet (translation invariant).
\end{Description}
\begin{Usage}
\begin{verbatim}
IFWT_TI(f_fft, psyJ_fft, lev, thr, nn, SOFT = FALSE)
\end{verbatim}
\end{Usage}
\begin{Arguments}
\begin{ldescription}
\item[\code{f\_fft}] vector of the  Fourier coefficient of $f$
\item[\code{psyJ\_fft}] vector of the  Fourier coefficient of the Meyer wavelet.
\item[\code{lev}] resolution level 
\item[\code{thr}] threshold (has lentgh=1)
\item[\code{nn}] sample size 
\item[\code{SOFT}] if SOFT=TRUE, uses the soft thresholding policy 
as opposed to the        hard (SOFT=FALSE, the default).  
\end{ldescription}
\end{Arguments}
\begin{Value}
Inverse Forward Wavelet Transform of a signal $f$, after thresholding.
\end{Value}
\begin{Author}\relax
Marc Raimondo
\end{Author}
\begin{References}\relax
Raimondo, M. and Stewart, M. (2007),
"The WaveD Transform in R", Journal of Statistical Software.
\end{References}
\begin{SeeAlso}\relax
\code{\LinkA{WaveD}{WaveD}}, ~~~
\end{SeeAlso}
\begin{Examples}
\begin{ExampleCode}
psyJ_fft=wavelet_YM(4,10,3);
f_fft=fft(sin(2*pi*seq(0,1,le=1024)));
IFWT_TI(f_fft, psyJ_fft, 4, 0, 1024)
\end{ExampleCode}
\end{Examples}

