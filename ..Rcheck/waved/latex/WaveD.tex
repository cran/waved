\HeaderA{WaveD}{WaveD}{WaveD}
\keyword{nonparametric}{WaveD}
\begin{Description}\relax
Performs statistical wavelet deconvolution using Meyer wavelet.
\end{Description}
\begin{Usage}
\begin{verbatim}
WaveD(yobs, g = c(1, rep(0, (length(yobs) - 1))), MC = FALSE, SOFT = FALSE, F = find.j1(g, scale(yobs))[2], L = 3, deg = 3, eta = sqrt(6), thr = maxithresh(yobs, g, eta = eta), label = "WaveD")
\end{verbatim}
\end{Usage}
\begin{Arguments}
\begin{ldescription}
\item[\code{yobs}] Sample of $f*g$ + (Gaussian noise), a vector of dyadic length 
(i.e. $2^(J-1)$ where J is the largest resolution level). 
Here f is the target function, g is the convolution kernel.
\item[\code{g}] Sample of g or g + (Gaussian noise), same length as yobs.
The default is the Dirac mass at 0.
\item[\code{MC}] Option to only return the (fast) translation-invariant WaveD estimate
(MC=TRUE) as opposed to the full WaveD output (MC=FALSE, the default), 
as described below. MC=TRUE recommended for Monte Carlo simulation.
\item[\code{SOFT}] if SOFT=TRUE, uses the soft thresholding policy as opposed to the
hard (SOFT=FALSE, the default).
\item[\code{F}] Finest resolution level; the default is the data-driven choice j1
(see Value below).
\item[\code{L}] Lowest resolution level; the default is 3.
\item[\code{deg}] The degree of the Meyer wavelet, either 1, 2, or 3 (the default).
\item[\code{eta}] Tuning parameter of the maxiset threshold; default is $\sqrt(6)$.
\item[\code{thr}] A vector of length $F-L+1$, giving thresholds at each resolution levels L,L+1,...,F; default is maxiset threshold.
\item[\code{label}] Auxiliary plotting parameter; do not change this.
\end{ldescription}
\end{Arguments}
\begin{Value}
In the case that MC=TRUE, WaveD returns a vector consisting of the translation-invariant WaveD estimate.
In the case that MC=FALSE (the default), WaveD returns a list with components
\begin{ldescription}
\item[\code{waved}] translation invariant WaveD transform; in the case MC=TRUE this is all that is returned.
\item[\code{ordinary}] ordinary WaveD transform
\item[\code{FWaveD}] Forward WaveD Transform; see \code{\LinkA{FWaveD}{FWaveD}}.
\item[\code{w}] alternate name for FWaveD
\item[\code{w.thr}] thresholded version of w
\item[\code{IWaveD}] Inverse WaveD Transform
\item[\code{iw}] alternate name for IWaveD
\item[\code{s}] estimate of the noise standard deviation
\item[\code{j1}] estimate of optimal resolution level (for maxiset threshold).
\item[\code{F}] Fine resolution level used (may be different to j1).
\item[\code{M}] estimate of optimal Fourier frequency (for maxiset threshold).
\item[\code{thr}] vector of thresholds used (default is maxiset threshold).
\item[\code{percent}] percentage of thresholding per resolution level
\item[\code{noise}] noise proxy, wavelet coefficients of the raw data at the largest resolution level, used for estimating noise features.
\item[\code{ps}] P-value of the Shapiro-Wilk test for normality applied to the noise proxy.
\item[\code{residuals}] wavelet coefficients that have been removed before fine level F.
\end{ldescription}
\end{Value}
\begin{Author}\relax
Marc Raimondo and Michael Stewart
\end{Author}
\begin{References}\relax
Cavalier, L. and Raimondo, M.  (2007), `Wavelet deconvolution with noisy eigen-values', {\em IEEE Trans. Signal
Process}, Vol. 55(6), In the press.

Donoho, D. and Raimondo, M.  (2004),
`Translation invariant deconvolution in a periodic setting', {\em The
International Journal of Wavelets, Multiresolution and Information
Processing} {\bf 14}(1),~415--423.

Johnstone, I., Kerkyacharian, G., Picard, D. and Raimondo, M.  (2004), 
`Wavelet deconvolution in a periodic
setting', {\em Journal of the Royal Statistical Society, Series B} {\bf
66}(3),~547--573.  with discussion pp.627-652.

Raimondo, M. and Stewart, M. (2007),
`The WaveD Transform in R', Journal of Statistical Software.
\end{References}
\begin{SeeAlso}\relax
\code{\LinkA{FWaveD}{FWaveD}}
\end{SeeAlso}
\begin{Examples}
\begin{ExampleCode}
library(waved)
data=waved.example(TRUE,FALSE)
doppler.wvd=WaveD(data$doppler.noisy,data$g)
summary(doppler.wvd)
\end{ExampleCode}
\end{Examples}

