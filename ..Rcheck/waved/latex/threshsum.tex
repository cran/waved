\HeaderA{threshsum}{Show threshold effects}{threshsum}
\keyword{internal}{threshsum}
\begin{Description}\relax
Provide a summary of threshold effects in WaveD fit.
\end{Description}
\begin{Usage}
\begin{verbatim}
threshsum(w.res, L = 3, F = (log2(length(w.res)) - 1))
\end{verbatim}
\end{Usage}
\begin{Arguments}
\begin{ldescription}
\item[\code{w.res}] A vector of wavelet coefficients 
\item[\code{L}] Low resolution level 
\item[\code{F}] Fine resolution level
\end{ldescription}
\end{Arguments}
\begin{Value}
A vector of length F-L+1, 
with ONES and ZEROS. The ZEROS show that
no coefficient remains at the corresponding resolution level.
\end{Value}
\begin{Author}\relax
Marc Raimondo
\end{Author}
\begin{References}\relax
Raimondo, M. and Stewart, M. (2007),
"The WaveD Transform in R", Journal of Statistical Software.
\end{References}
\begin{SeeAlso}\relax
\code{\LinkA{WaveD}{WaveD}}
\end{SeeAlso}
\begin{Examples}
\begin{ExampleCode}
library(waved)
data=waved.example(TRUE,FALSE)
doppler.wvd=WaveD(data$doppler.noisy,data$g)
threshsum(doppler.wvd$w,3,8)

\end{ExampleCode}
\end{Examples}

