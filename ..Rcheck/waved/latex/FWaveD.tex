\HeaderA{FWaveD}{FWaveD}{FWaveD}
\keyword{nonparametric}{FWaveD}
\begin{Description}\relax
Computes the Forward WaveD Transform.
\end{Description}
\begin{Usage}
\begin{verbatim}
FWaveD(y, g = 1, L = 3, deg = 3, F = (log2(length(y)) - 1), thr = rep(0, log2(length(y))), SOFT = FALSE)
\end{verbatim}
\end{Usage}
\begin{Arguments}
\begin{ldescription}
\item[\code{y}] Sample of $f*g$ + (Gaussian noise), a vector of dyadic length 
(i.e. $2^(J-1)$ where J is the largest resolution level). 
Here f is the target function, g is the convolution kernel.
\item[\code{g}] Sample of g or g + (Gaussian noise), same length as yobs.
The default is the Dirac mass at 0.
\item[\code{L}] Lowest resolution level; the default is 3.
\item[\code{deg}] The degree of the Meyer wavelet, either 1, 2, or 3 (the default).
\item[\code{F}] Finest resolution level; the default is the data-driven choice j1
(see Value below).
\item[\code{thr}] A vector of length $F-L+1$, giving thresholds at each resolution levels L,L+1,...,F; default is maxiset threshold.
\item[\code{SOFT}] if SOFT=TRUE, uses the soft thresholding policy as opposed to the
hard (SOFT=FALSE, the default).
\end{ldescription}
\end{Arguments}
\begin{Value}
Returns a vector of wavelet coefficients of length n (the same length as y),
the last n/2 entries are wavelet coefficients at resolution level $J-1$, where
$J = log_2(n)$; the $n/4$ entries before that are the wavelet coefficients at
resolution level $J-2$, and so on until level L. In addition the $2^L$ entries
are scaling coefficients at coarse level C=L.
\end{Value}
\begin{References}\relax
Johnstone, I., Kerkyacharian, G., Picard, D. and Raimondo, M.  (2004), 
`Wavelet deconvolution in a periodic
setting', {\em Journal of the Royal Statistical Society, Series B} {\bf
66}(3),~547--573.  with discussion pp.627-652.

Raimondo, M. and Stewart, M. (2006),
`The WaveD Transform in R', preprint, School and Mathematics and Statistics,
University of Sydney.
\end{References}
\begin{SeeAlso}\relax
\code{\LinkA{WaveD}{WaveD}}
\end{SeeAlso}
\begin{Examples}
\begin{ExampleCode}
library(waved)
data=waved.example(TRUE,FALSE)
lidar.w=FWaveD(data$lidar.blur,data$g)
\end{ExampleCode}
\end{Examples}

