\HeaderA{MultiThresh1}{Maxiset threshold}{MultiThresh1}
\keyword{internal}{MultiThresh1}
\begin{Description}\relax
Compute the maxiset threshold for WaveD fit.
\end{Description}
\begin{Usage}
\begin{verbatim}
MultiThresh1(s, g, L, eta)
\end{verbatim}
\end{Usage}
\begin{Arguments}
\begin{ldescription}
\item[\code{s}] noise standard deviation  
\item[\code{g}] Sample of g or g + (Gaussian noise). 
\item[\code{L}] Lowest resolution level. 
\item[\code{eta}] Tuning parameter of the maxiset threshold. 
\end{ldescription}
\end{Arguments}
\begin{Value}
vector of thresholds
\end{Value}
\begin{Author}\relax
Marc Raimondo
\end{Author}
\begin{References}\relax
Raimondo, M. and Stewart, M. (2007),
"The WaveD Transform in R", Journal of Statistical Software.
\end{References}
\begin{SeeAlso}\relax
\code{\LinkA{WaveD}{WaveD}},
\end{SeeAlso}
\begin{Examples}
\begin{ExampleCode}

MultiThresh1(.1, sin(2*pi*seq(0,1,le=1024)), 3, sqrt(2))
\end{ExampleCode}
\end{Examples}

