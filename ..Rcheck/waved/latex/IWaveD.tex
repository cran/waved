\HeaderA{IWaveD}{Computes the Inverse WaveD transform}{IWaveD}
\keyword{internal}{IWaveD}
\begin{Description}\relax
Computes the Inverse WaveD transform
based on a vector of wavelet coefficients.
\end{Description}
\begin{Usage}
\begin{verbatim}
IWaveD(w, C = 3, deg = 3, F = log2(length(w)))
\end{verbatim}
\end{Usage}
\begin{Arguments}
\begin{ldescription}
\item[\code{w}] vector of wavelet coefficents, must be of dyadic length; typically returned by the function \code{\LinkA{FWaveD}{FWaveD}}
\item[\code{C}] coarse resolution level
\item[\code{deg}] degree of the Meyer wavelet
\item[\code{F}] fine resolution level
\end{ldescription}
\end{Arguments}
\begin{Value}
Returns a vector of the same length as w, giving the inverse wavelet transform.
\end{Value}
\begin{Author}\relax
Marc Raimondo
\end{Author}
\begin{References}\relax
Johnstone, I., Kerkyacharian, G., Picard, D. and Raimondo, M.  (2004), 
`Wavelet deconvolution in a periodic
setting', {\em Journal of the Royal Statistical Society, Series B} {\bf
66}(3),~547--573.  with discussion pp.627-652.

Raimondo, M. and Stewart, M. (2006),
`The WaveD Transform in R', preprint, School and Mathematics and Statistics,
University of Sydney.
\end{References}
\begin{SeeAlso}\relax
\code{\LinkA{WaveD}{WaveD}}
\end{SeeAlso}
\begin{Examples}
\begin{ExampleCode}
library(waved)
data=waved.example(TRUE,FALSE)
lidar.w=FWaveD(data$lidar.blur,data$g)  # lidar.blur is lidar*g 
IWaveD(lidar.w)               # same as lidar
\end{ExampleCode}
\end{Examples}

