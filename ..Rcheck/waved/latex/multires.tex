\HeaderA{multires}{Create Multi-resolution Plot}{multires}
\keyword{internal}{multires}
\begin{Description}\relax
Depicts wavelet coefficients according to time and resolution level.
\end{Description}
\begin{Usage}
\begin{verbatim}
multires(wcUntrimmed, lowest = 3, coarse = 3, highestplot = NULL, descending = FALSE, sc = 1)
\end{verbatim}
\end{Usage}
\begin{Arguments}
\begin{ldescription}
\item[\code{wcUntrimmed}] Vector of wavelet coefficients; must be of dyadic length.
\item[\code{lowest}] Lowest resolution level.
\item[\code{coarse}] Coarse resolution level; same as lowest.
\item[\code{highestplot}] Highest resolution level.
\item[\code{descending}] logical indicating whether resolutions are depicted with highest at the top of the plot (FALSE, the default), or at the bottom (TRUE).
\item[\code{sc}] graphical scaling parameter of heights of lines representing wavelet coefficients; default is 1.
\end{ldescription}
\end{Arguments}
\begin{Value}
Depicts wavelet coefficients according to time (horizontal axis)
and resolution level lowest, lowest+1,...,highestplot.
\end{Value}
\begin{Author}\relax
Marc Raimondo and Michael Stewart
\end{Author}
\begin{References}\relax
Donoho, D. and Raimondo, M.  (2004),
`Translation invariant deconvolution in a periodic setting', {\em The
International Journal of Wavelets, Multiresolution and Information
Processing} {\bf 14}(1),~415--423.

Johnstone, I., Kerkyacharian, G., Picard, D. and Raimondo, M.  (2004), 
`Wavelet deconvolution in a periodic
setting', {\em Journal of the Royal Statistical Society, Series B} {\bf
66}(3),~547--573.  with discussion pp.627-652.

Raimondo, M. and Stewart, M. (2006),
`The WaveD Transform in R', preprint, School and Mathematics and Statistics,
University of Sydney.
\end{References}
\begin{SeeAlso}\relax
\code{\LinkA{WaveD}{WaveD}}
\end{SeeAlso}
\begin{Examples}
\begin{ExampleCode}
library(waved)
data=waved.example(TRUE,FALSE)
lidar.w=FWaveD(data$lidar.blur,data$g,F=7)
multires(lidar.w,lo=3,hi=7)
\end{ExampleCode}
\end{Examples}

