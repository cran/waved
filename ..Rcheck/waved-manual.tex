\documentclass{article}
\usepackage[ae,hyper]{Rd}
\begin{document}
\HeaderA{BlurSignal}{Blurr Signal}{BlurSignal}
\keyword{internal}{BlurSignal}
\begin{Description}\relax
Compute the convolution of $f$ and $g$ in the periodic setting.
\end{Description}
\begin{Usage}
\begin{verbatim}
BlurSignal(f, g)
\end{verbatim}
\end{Usage}
\begin{Arguments}
\begin{ldescription}
\item[\code{f}] A sample of $f$.
\item[\code{g}] A sampel of $g$ 
\end{ldescription}
\end{Arguments}
\begin{Value}
Returns the convolution of $f$ and $g$.
\end{Value}
\begin{Author}\relax
Marc Raimondo
\end{Author}
\begin{References}\relax
Raimondo, M. and Stewart, M. (2007),
"The WaveD Transform in R", Journal of Statistical Software.
\end{References}
\begin{SeeAlso}\relax
\code{\LinkA{WaveD}{WaveD}}
\end{SeeAlso}
\begin{Examples}
\begin{ExampleCode} 
x=1:10
y=1:10 
BlurSignal(x,y)

\end{ExampleCode}
\end{Examples}

\HeaderA{FWT\_TI}{Forward Wavelet Transform (translation invariant).}{FWT.Rul.TI}
\keyword{internal}{FWT\_TI}
\begin{Description}\relax
Compute the Forward Wavelet Transform of a signal $f$ for the Meyer wavelet (translation invariant).
\end{Description}
\begin{Usage}
\begin{verbatim}
FWT_TI(f_fft, psyJ_fft)
\end{verbatim}
\end{Usage}
\begin{Arguments}
\begin{ldescription}
\item[\code{f\_fft}] vector of the  Fourier coefficient of $f$.
\item[\code{psyJ\_fft}] vector of the  Fourier coefficient of the Meyer wavelet.
\end{ldescription}
\end{Arguments}
\begin{Value}
vector of wavelet coefficients (non-ordered).
\end{Value}
\begin{Author}\relax
Marc Raimondo
\end{Author}
\begin{References}\relax
Raimondo, M. and Stewart, M. (2007),
"The WaveD Transform in R", Journal of Statistical Software.
\end{References}
\begin{SeeAlso}\relax
\code{\LinkA{WaveD}{WaveD}}
\end{SeeAlso}
\begin{Examples}
\begin{ExampleCode}
 psyJ_fft=wavelet_YM(4,10,3);
 f_fft=fft(sin(2*pi*seq(0,1,le=1024)))
 FWT_TI(f_fft,psyJ_fft)
 \end{ExampleCode}
\end{Examples}

\HeaderA{FWaveD}{FWaveD}{FWaveD}
\keyword{nonparametric}{FWaveD}
\begin{Description}\relax
Computes the Forward WaveD Transform.
\end{Description}
\begin{Usage}
\begin{verbatim}
FWaveD(y, g = 1, L = 3, deg = 3, F = (log2(length(y)) - 1), thr = rep(0, log2(length(y))), SOFT = FALSE)
\end{verbatim}
\end{Usage}
\begin{Arguments}
\begin{ldescription}
\item[\code{y}] Sample of $f*g$ + (Gaussian noise), a vector of dyadic length 
(i.e. $2^(J-1)$ where J is the largest resolution level). 
Here f is the target function, g is the convolution kernel.
\item[\code{g}] Sample of g or g + (Gaussian noise), same length as yobs.
The default is the Dirac mass at 0.
\item[\code{L}] Lowest resolution level; the default is 3.
\item[\code{deg}] The degree of the Meyer wavelet, either 1, 2, or 3 (the default).
\item[\code{F}] Finest resolution level; the default is the data-driven choice j1
(see Value below).
\item[\code{thr}] A vector of length $F-L+1$, giving thresholds at each resolution levels L,L+1,...,F; default is maxiset threshold.
\item[\code{SOFT}] if SOFT=TRUE, uses the soft thresholding policy as opposed to the
hard (SOFT=FALSE, the default).
\end{ldescription}
\end{Arguments}
\begin{Value}
Returns a vector of wavelet coefficients of length n (the same length as y),
the last n/2 entries are wavelet coefficients at resolution level $J-1$, where
$J = log_2(n)$; the $n/4$ entries before that are the wavelet coefficients at
resolution level $J-2$, and so on until level L. In addition the $2^L$ entries
are scaling coefficients at coarse level C=L.
\end{Value}
\begin{References}\relax
Johnstone, I., Kerkyacharian, G., Picard, D. and Raimondo, M.  (2004), 
`Wavelet deconvolution in a periodic
setting', {\em Journal of the Royal Statistical Society, Series B} {\bf
66}(3),~547--573.  with discussion pp.627-652.

Raimondo, M. and Stewart, M. (2006),
`The WaveD Transform in R', preprint, School and Mathematics and Statistics,
University of Sydney.
\end{References}
\begin{SeeAlso}\relax
\code{\LinkA{WaveD}{WaveD}}
\end{SeeAlso}
\begin{Examples}
\begin{ExampleCode}
library(waved)
data=waved.example(TRUE,FALSE)
lidar.w=FWaveD(data$lidar.blur,data$g)
\end{ExampleCode}
\end{Examples}

\HeaderA{HardThresh}{Hard Threshold}{HardThresh}
\keyword{internal}{HardThresh}
\begin{Description}\relax
Apply hard threshold.
\end{Description}
\begin{Usage}
\begin{verbatim}
HardThresh(y, t)
\end{verbatim}
\end{Usage}
\begin{Arguments}
\begin{ldescription}
\item[\code{y}] vector
\item[\code{t}] threshold 
\end{ldescription}
\end{Arguments}
\begin{Value}
vector $y$ thresholded:  entries below $t$ are replaced by zeros.
\end{Value}
\begin{Author}\relax
Marc Raimondo
\end{Author}
\begin{References}\relax
Raimondo, M. and Stewart, M. (2007),
"The WaveD Transform in R", Journal of Statistical Software.
\end{References}
\begin{SeeAlso}\relax
\code{\LinkA{WaveD}{WaveD}}
\end{SeeAlso}
\begin{Examples}
\begin{ExampleCode}
HardThresh(1:5,2)
  \end{ExampleCode}
\end{Examples}

\HeaderA{IFWT\_TI}{Inverse Forward Wavelet Transform (translation invariant).}{IFWT.Rul.TI}
\keyword{internal}{IFWT\_TI}
\begin{Description}\relax
Compute the Inverse Forward Wavelet Transform of a signal $f$ for the Meyer wavelet (translation invariant).
\end{Description}
\begin{Usage}
\begin{verbatim}
IFWT_TI(f_fft, psyJ_fft, lev, thr, nn, SOFT = FALSE)
\end{verbatim}
\end{Usage}
\begin{Arguments}
\begin{ldescription}
\item[\code{f\_fft}] vector of the  Fourier coefficient of $f$
\item[\code{psyJ\_fft}] vector of the  Fourier coefficient of the Meyer wavelet.
\item[\code{lev}] resolution level 
\item[\code{thr}] threshold (has lentgh=1)
\item[\code{nn}] sample size 
\item[\code{SOFT}] if SOFT=TRUE, uses the soft thresholding policy 
as opposed to the        hard (SOFT=FALSE, the default).  
\end{ldescription}
\end{Arguments}
\begin{Value}
Inverse Forward Wavelet Transform of a signal $f$, after thresholding.
\end{Value}
\begin{Author}\relax
Marc Raimondo
\end{Author}
\begin{References}\relax
Raimondo, M. and Stewart, M. (2007),
"The WaveD Transform in R", Journal of Statistical Software.
\end{References}
\begin{SeeAlso}\relax
\code{\LinkA{WaveD}{WaveD}}, ~~~
\end{SeeAlso}
\begin{Examples}
\begin{ExampleCode}
psyJ_fft=wavelet_YM(4,10,3);
f_fft=fft(sin(2*pi*seq(0,1,le=1024)));
IFWT_TI(f_fft, psyJ_fft, 4, 0, 1024)
\end{ExampleCode}
\end{Examples}

\HeaderA{IWaveD}{Computes the Inverse WaveD transform}{IWaveD}
\keyword{internal}{IWaveD}
\begin{Description}\relax
Computes the Inverse WaveD transform
based on a vector of wavelet coefficients.
\end{Description}
\begin{Usage}
\begin{verbatim}
IWaveD(w, C = 3, deg = 3, F = log2(length(w)))
\end{verbatim}
\end{Usage}
\begin{Arguments}
\begin{ldescription}
\item[\code{w}] vector of wavelet coefficents, must be of dyadic length; typically returned by the function \code{\LinkA{FWaveD}{FWaveD}}
\item[\code{C}] coarse resolution level
\item[\code{deg}] degree of the Meyer wavelet
\item[\code{F}] fine resolution level
\end{ldescription}
\end{Arguments}
\begin{Value}
Returns a vector of the same length as w, giving the inverse wavelet transform.
\end{Value}
\begin{Author}\relax
Marc Raimondo
\end{Author}
\begin{References}\relax
Johnstone, I., Kerkyacharian, G., Picard, D. and Raimondo, M.  (2004), 
`Wavelet deconvolution in a periodic
setting', {\em Journal of the Royal Statistical Society, Series B} {\bf
66}(3),~547--573.  with discussion pp.627-652.

Raimondo, M. and Stewart, M. (2006),
`The WaveD Transform in R', preprint, School and Mathematics and Statistics,
University of Sydney.
\end{References}
\begin{SeeAlso}\relax
\code{\LinkA{WaveD}{WaveD}}
\end{SeeAlso}
\begin{Examples}
\begin{ExampleCode}
library(waved)
data=waved.example(TRUE,FALSE)
lidar.w=FWaveD(data$lidar.blur,data$g)  # lidar.blur is lidar*g 
IWaveD(lidar.w)               # same as lidar
\end{ExampleCode}
\end{Examples}

\HeaderA{MeyerWindow}{Meyer wavelet window}{MeyerWindow}
\keyword{internal}{MeyerWindow}
\begin{Description}\relax
Auxiliary window function for Meyer wavelets.
\end{Description}
\begin{Usage}
\begin{verbatim}
MeyerWindow(xi, deg)
\end{verbatim}
\end{Usage}
\begin{Arguments}
\begin{ldescription}
\item[\code{xi}] Abscissa values for window evaluation
\item[\code{deg}] The degree of the Meyer wavelet, either 1, 2, or 3
\end{ldescription}

\value{a sampel vector of the window function for Meyer wavelets.}
\end{Arguments}
\begin{Value}
a sampel vector of the window function for Meyer wavelets.
\end{Value}
\begin{Author}\relax
Marc Raimondo
\end{Author}
\begin{References}\relax
Raimondo, M. and Stewart, M. (2007),
"The WaveD Transform in R", Journal of Statistical Software.
\end{References}
\begin{SeeAlso}\relax
\code{\LinkA{WaveD}{WaveD}}
\end{SeeAlso}
\begin{Examples}
\begin{ExampleCode}
  plot(seq(0,1,le=1000),MeyerWindow(seq(0,1,le=1000),3),type='l')
\end{ExampleCode}
\end{Examples}

\HeaderA{MultiThresh1}{Maxiset threshold}{MultiThresh1}
\keyword{internal}{MultiThresh1}
\begin{Description}\relax
Compute the maxiset threshold for WaveD fit.
\end{Description}
\begin{Usage}
\begin{verbatim}
MultiThresh1(s, g, L, eta)
\end{verbatim}
\end{Usage}
\begin{Arguments}
\begin{ldescription}
\item[\code{s}] noise standard deviation  
\item[\code{g}] Sample of g or g + (Gaussian noise). 
\item[\code{L}] Lowest resolution level. 
\item[\code{eta}] Tuning parameter of the maxiset threshold. 
\end{ldescription}
\end{Arguments}
\begin{Value}
vector of thresholds
\end{Value}
\begin{Author}\relax
Marc Raimondo
\end{Author}
\begin{References}\relax
Raimondo, M. and Stewart, M. (2007),
"The WaveD Transform in R", Journal of Statistical Software.
\end{References}
\begin{SeeAlso}\relax
\code{\LinkA{WaveD}{WaveD}},
\end{SeeAlso}
\begin{Examples}
\begin{ExampleCode}

MultiThresh1(.1, sin(2*pi*seq(0,1,le=1024)), 3, sqrt(2))
\end{ExampleCode}
\end{Examples}

\HeaderA{PhaseC}{Phase matrix}{PhaseC}
\keyword{internal}{PhaseC}
\begin{Description}\relax
compute phase matrix, auxilliary function compute wavelet coefficients
in the Fourier Domain
\end{Description}
\begin{Usage}
\begin{verbatim}
PhaseC(l, j)
\end{verbatim}
\end{Usage}
\begin{Arguments}
\begin{ldescription}
\item[\code{l}] Fourier frequency (integer) 
\item[\code{j}] resolution level (integer) 
\end{ldescription}
\end{Arguments}
\begin{Value}
Matrix of phases.
\end{Value}
\begin{Author}\relax
Marc Raimondo
\end{Author}
\begin{References}\relax
Raimondo, M. and Stewart, M. (2007),
"The WaveD Transform in R", Journal of Statistical Software.
\end{References}
\begin{SeeAlso}\relax
\code{\LinkA{WaveD}{WaveD}}
\end{SeeAlso}
\begin{Examples}
\begin{ExampleCode}
PhaseC(3,4)
\end{ExampleCode}
\end{Examples}

\HeaderA{SoftThresh}{Soft Threshold}{SoftThresh}
\keyword{internal}{SoftThresh}
\begin{Description}\relax
Apply soft threshold.
\end{Description}
\begin{Usage}
\begin{verbatim}
SoftThresh(y, t)
\end{verbatim}
\end{Usage}
\begin{Arguments}
\begin{ldescription}
\item[\code{y}] vector
\item[\code{t}] threshold 
\end{ldescription}
\end{Arguments}
\begin{Value}
vector $y$ thresholded:  entries below $t$ are replaced by shrunken versions.
\end{Value}
\begin{Author}\relax
Marc Raimondo
\end{Author}
\begin{References}\relax
Raimondo, M. and Stewart, M. (2007),
"The WaveD Transform in R", Journal of Statistical Software.
\end{References}
\begin{SeeAlso}\relax
\code{\LinkA{WaveD}{WaveD}}
\end{SeeAlso}
\begin{Examples}
\begin{ExampleCode}
SoftThresh(1:5,2)
  \end{ExampleCode}
\end{Examples}

\HeaderA{WaveD}{WaveD}{WaveD}
\keyword{nonparametric}{WaveD}
\begin{Description}\relax
Performs statistical wavelet deconvolution using Meyer wavelet.
\end{Description}
\begin{Usage}
\begin{verbatim}
WaveD(yobs, g = c(1, rep(0, (length(yobs) - 1))), MC = FALSE, SOFT = FALSE, F = find.j1(g, scale(yobs))[2], L = 3, deg = 3, eta = sqrt(6), thr = maxithresh(yobs, g, eta = eta), label = "WaveD")
\end{verbatim}
\end{Usage}
\begin{Arguments}
\begin{ldescription}
\item[\code{yobs}] Sample of $f*g$ + (Gaussian noise), a vector of dyadic length 
(i.e. $2^(J-1)$ where J is the largest resolution level). 
Here f is the target function, g is the convolution kernel.
\item[\code{g}] Sample of g or g + (Gaussian noise), same length as yobs.
The default is the Dirac mass at 0.
\item[\code{MC}] Option to only return the (fast) translation-invariant WaveD estimate
(MC=TRUE) as opposed to the full WaveD output (MC=FALSE, the default), 
as described below. MC=TRUE recommended for Monte Carlo simulation.
\item[\code{SOFT}] if SOFT=TRUE, uses the soft thresholding policy as opposed to the
hard (SOFT=FALSE, the default).
\item[\code{F}] Finest resolution level; the default is the data-driven choice j1
(see Value below).
\item[\code{L}] Lowest resolution level; the default is 3.
\item[\code{deg}] The degree of the Meyer wavelet, either 1, 2, or 3 (the default).
\item[\code{eta}] Tuning parameter of the maxiset threshold; default is $\sqrt(6)$.
\item[\code{thr}] A vector of length $F-L+1$, giving thresholds at each resolution levels L,L+1,...,F; default is maxiset threshold.
\item[\code{label}] Auxiliary plotting parameter; do not change this.
\end{ldescription}
\end{Arguments}
\begin{Value}
In the case that MC=TRUE, WaveD returns a vector consisting of the translation-invariant WaveD estimate.
In the case that MC=FALSE (the default), WaveD returns a list with components
\begin{ldescription}
\item[\code{waved}] translation invariant WaveD transform; in the case MC=TRUE this is all that is returned.
\item[\code{ordinary}] ordinary WaveD transform
\item[\code{FWaveD}] Forward WaveD Transform; see \code{\LinkA{FWaveD}{FWaveD}}.
\item[\code{w}] alternate name for FWaveD
\item[\code{w.thr}] thresholded version of w
\item[\code{IWaveD}] Inverse WaveD Transform
\item[\code{iw}] alternate name for IWaveD
\item[\code{s}] estimate of the noise standard deviation
\item[\code{j1}] estimate of optimal resolution level (for maxiset threshold).
\item[\code{F}] Fine resolution level used (may be different to j1).
\item[\code{M}] estimate of optimal Fourier frequency (for maxiset threshold).
\item[\code{thr}] vector of thresholds used (default is maxiset threshold).
\item[\code{percent}] percentage of thresholding per resolution level
\item[\code{noise}] noise proxy, wavelet coefficients of the raw data at the largest resolution level, used for estimating noise features.
\item[\code{ps}] P-value of the Shapiro-Wilk test for normality applied to the noise proxy.
\item[\code{residuals}] wavelet coefficients that have been removed before fine level F.
\end{ldescription}
\end{Value}
\begin{Author}\relax
Marc Raimondo and Michael Stewart
\end{Author}
\begin{References}\relax
Cavalier, L. and Raimondo, M.  (2007), `Wavelet deconvolution with noisy eigen-values', {\em IEEE Trans. Signal
Process}, Vol. 55(6), In the press.

Donoho, D. and Raimondo, M.  (2004),
`Translation invariant deconvolution in a periodic setting', {\em The
International Journal of Wavelets, Multiresolution and Information
Processing} {\bf 14}(1),~415--423.

Johnstone, I., Kerkyacharian, G., Picard, D. and Raimondo, M.  (2004), 
`Wavelet deconvolution in a periodic
setting', {\em Journal of the Royal Statistical Society, Series B} {\bf
66}(3),~547--573.  with discussion pp.627-652.

Raimondo, M. and Stewart, M. (2007),
`The WaveD Transform in R', Journal of Statistical Software.
\end{References}
\begin{SeeAlso}\relax
\code{\LinkA{FWaveD}{FWaveD}}
\end{SeeAlso}
\begin{Examples}
\begin{ExampleCode}
library(waved)
data=waved.example(TRUE,FALSE)
doppler.wvd=WaveD(data$doppler.noisy,data$g)
summary(doppler.wvd)
\end{ExampleCode}
\end{Examples}

\HeaderA{WaveDjC}{WaveD projection, coarse level.}{WaveDjC}
\keyword{internal}{WaveDjC}
\begin{Description}\relax
Compute WaveD projection of $f$, coarse level.
\end{Description}
\begin{Usage}
\begin{verbatim}
WaveDjC(y_fft, f2fft, j)
\end{verbatim}
\end{Usage}
\begin{Arguments}
\begin{ldescription}
\item[\code{y\_fft}] Fourier transform of $f$. 
\item[\code{f2fft}] Fourier transform of the wavelet.
\item[\code{j}] Resolution level. 
\end{ldescription}
\end{Arguments}
\begin{Value}
Vector:  WaveD projection of $f$, coarse resolution level.
\end{Value}
\begin{Author}\relax
Marc Raimondo
\end{Author}
\begin{References}\relax
Raimondo, M. and Stewart, M. (2007),
"The WaveD Transform in R", Journal of Statistical Software.
\end{References}
\begin{SeeAlso}\relax
\code{\LinkA{WaveD}{WaveD}}
\end{SeeAlso}
\begin{Examples}
\begin{ExampleCode}

waveJ0_fft=scaling_YM(3,10,3);
WaveDjC(fft(sin(2*pi*seq(0,1,le=1024))),waveJ0_fft,3)
\end{ExampleCode}
\end{Examples}

\HeaderA{WaveDjD}{WaveD projection, details.}{WaveDjD}
\keyword{internal}{WaveDjD}
\begin{Description}\relax
Compute WaveD projection of $f$, details.
\end{Description}
\begin{Usage}
\begin{verbatim}
WaveDjD(y_fft, f2fft, j)
\end{verbatim}
\end{Usage}
\begin{Arguments}
\begin{ldescription}
\item[\code{y\_fft}] Fourier transform of $f$. 
\item[\code{f2fft}] Fourier transform of the wavelet.
\item[\code{j}] Resolution level. 
\end{ldescription}
\end{Arguments}
\begin{Value}
Vector:  WaveD projection of $f$, details.
\end{Value}
\begin{Author}\relax
Marc Raimondo
\end{Author}
\begin{References}\relax
Raimondo, M. and Stewart, M. (2007),
"The WaveD Transform in R", Journal of Statistical Software.
\end{References}
\begin{SeeAlso}\relax
\code{\LinkA{WaveD}{WaveD}}
\end{SeeAlso}
\begin{Examples}
\begin{ExampleCode}

waveJ0_fft=wavelet_YM(5,10,3);
WaveDjD(fft(sin(2*pi*seq(0,1,le=1024))),waveJ0_fft,3)
\end{ExampleCode}
\end{Examples}

\HeaderA{WaveDjF}{WaveD projection, fine resolution level.}{WaveDjF}
\keyword{internal}{WaveDjF}
\begin{Description}\relax
Compute WaveD projection of $f$, fine resolution level.
\end{Description}
\begin{Usage}
\begin{verbatim}
WaveDjF(f1fft, f2fft, j)
\end{verbatim}
\end{Usage}
\begin{Arguments}
\begin{ldescription}
\item[\code{y\_fft}] Fourier transform of $f$. 
\item[\code{f2fft}] Fourier transform of the wavelet.
\item[\code{j}] Resolution level. 
\end{ldescription}
\end{Arguments}
\begin{Value}
Vector:  WaveD projection of $f$, fine resolution level.
\end{Value}
\begin{Author}\relax
Marc Raimondo
\end{Author}
\begin{References}\relax
Raimondo, M. and Stewart, M. (2007),
"The WaveD Transform in R", Journal of Statistical Software.
\end{References}
\begin{SeeAlso}\relax
\code{\LinkA{WaveD}{WaveD}}
\end{SeeAlso}
\begin{Examples}
\begin{ExampleCode}

waveJ0_fft=fine_YM(9,10,3);
WaveDjF(fft(sin(2*pi*seq(0,1,le=1024))),waveJ0_fft,3)
\end{ExampleCode}
\end{Examples}

\HeaderA{dyad}{Dyadic band}{dyad}
\keyword{internal}{dyad}
\begin{Description}\relax
Returns a vector of integers $2^j+1,...,2^{j+1}$.
\end{Description}
\begin{Usage}
\begin{verbatim}
dyad(j)
\end{verbatim}
\end{Usage}
\begin{Arguments}
\begin{ldescription}
\item[\code{j}] Resolution Level
\end{ldescription}
\end{Arguments}
\begin{Value}
Returns a vector of integers $2^j+1,...,2^{j+1}$.
\end{Value}
\begin{Author}\relax
Marc Raimondo
\end{Author}
\begin{SeeAlso}\relax
\code{\LinkA{FWaveD}{FWaveD}}
\end{SeeAlso}
\begin{Examples}
\begin{ExampleCode}
library(waved)
data=waved.example(TRUE,FALSE)
lidar.w=FWaveD(data$lidar.blur,data$g)
lidar.w[dyad(7)]
\end{ExampleCode}
\end{Examples}

\HeaderA{dyadjk}{Lexicographic ordering (dyadic)}{dyadjk}
\keyword{internal}{dyadjk}
\begin{Description}\relax
return the index of a
wavelet coefficient using dyadic lexicographic ordering
\end{Description}
\begin{Usage}
\begin{verbatim}
dyadjk(j, k)
\end{verbatim}
\end{Usage}
\begin{Arguments}
\begin{ldescription}
\item[\code{j}] Resolution level (integer) 
\item[\code{k}] Location parameter (0,1,...,$2^j$-1)
\end{ldescription}
\end{Arguments}
\begin{Value}
Returns an integer giving the index position
of the wavelet coefficient $w_{j,k}$ in a vector of wavelet 
coefficients.
\end{Value}
\begin{Author}\relax
Marc Raimondo
\end{Author}
\begin{References}\relax
Raimondo, M. and Stewart, M. (2007),
`The WaveD Transform in R', Journal of Statistical Software.
\end{References}
\begin{SeeAlso}\relax
\code{\LinkA{FWaveD}{FWaveD} }
\end{SeeAlso}
\begin{Examples}
\begin{ExampleCode}
print(dyadjk(5,4))
  \end{ExampleCode}
\end{Examples}

\HeaderA{fftshift}{Shift Fourier frequencies}{fftshift}
\keyword{internal}{fftshift}
\begin{Description}\relax
rearranges the outputs of fft by moving 
the zero-frequency component to the center of the array
\end{Description}
\begin{Usage}
\begin{verbatim}
fftshift(y)
\end{verbatim}
\end{Usage}
\begin{Arguments}
\begin{ldescription}
\item[\code{y}] A vector
\end{ldescription}
\end{Arguments}
\begin{Value}
Rearranged version of the vector $y$
\end{Value}
\begin{Author}\relax
Marc Raimondo
\end{Author}
\begin{References}\relax
Raimondo, M. and Stewart, M. (2007),
`The WaveD Transform in R', Journal of Statistical Software.
\end{References}
\begin{SeeAlso}\relax
\code{\LinkA{fft}{fft}}
\end{SeeAlso}
\begin{Examples}
\begin{ExampleCode} print(fftshift(1:5))\end{ExampleCode}
\end{Examples}

\HeaderA{find.j1}{Fine  resolution level for WaveD fit}{find.j1}
\keyword{internal}{find.j1}
\begin{Description}\relax
Find the optimal Fourier frequency and resolution
level for WaveD fit
\end{Description}
\begin{Usage}
\begin{verbatim}
find.j1(g, sigma)
\end{verbatim}
\end{Usage}
\begin{Arguments}
\begin{ldescription}
\item[\code{g}] vector (convolution kenel)
\item[\code{sigma}] noise standard deviation 
\end{ldescription}
\end{Arguments}
\begin{Value}
\begin{ldescription}
\item[\code{M}] Fourier frequency
\item[\code{j1}] Resolution level
\end{ldescription}
\end{Value}
\begin{Author}\relax
Marc Raimondo
\end{Author}
\begin{References}\relax
Raimondo, M. and Stewart, M. (2007),
`The WaveD Transform in R', Journal of Statistical Software.
\end{References}
\begin{SeeAlso}\relax
\code{\LinkA{WaveD}{WaveD}}
\end{SeeAlso}
\begin{Examples}
\begin{ExampleCode}
library(waved)
data=waved.example(TRUE,FALSE)
find.j1(data$g,data$sigma)

\end{ExampleCode}
\end{Examples}

\HeaderA{findONE}{Find positive entries}{findONE}
\keyword{internal}{findONE}
\begin{Description}\relax
Find positive entries in a vector
\end{Description}
\begin{Usage}
\begin{verbatim}
findONE(x)
\end{verbatim}
\end{Usage}
\begin{Arguments}
\begin{ldescription}
\item[\code{x}] vector
\end{ldescription}
\end{Arguments}
\begin{Value}
A vector of indices where $x$
has positive values.
\end{Value}
\begin{Author}\relax
Marc Raimondo
\end{Author}
\begin{References}\relax
Raimondo, M. and Stewart, M. (2007),
"The WaveD Transform in R", Journal of Statistical Software.
\end{References}
\begin{SeeAlso}\relax
\code{\LinkA{findZERO}{findZERO}}
\end{SeeAlso}
\begin{Examples}
\begin{ExampleCode}findONE(-5:5)
\end{ExampleCode}
\end{Examples}

\HeaderA{findZERO}{Find negative entries}{findZERO}
\keyword{internal}{findZERO}
\begin{Description}\relax
Find negative entries in a vector
\end{Description}
\begin{Usage}
\begin{verbatim}
findZERO(x)
\end{verbatim}
\end{Usage}
\begin{Arguments}
\begin{ldescription}
\item[\code{x}] vector
\end{ldescription}
\end{Arguments}
\begin{Value}
A vector of indices where $x$
has negative values.
\end{Value}
\begin{Author}\relax
Marc Raimondo
\end{Author}
\begin{References}\relax
Raimondo, M. and Stewart, M. (2007),
"The WaveD Transform in R", Journal of Statistical Software.
\end{References}
\begin{SeeAlso}\relax
\code{\LinkA{findONE}{findONE}}
\end{SeeAlso}
\begin{Examples}
\begin{ExampleCode}
findZERO(-5:5)
\end{ExampleCode}
\end{Examples}

\HeaderA{fine\_YM}{~~function to do ... ~~}{fine.Rul.YM}
\keyword{internal}{fine\_YM}
\begin{Description}\relax
generate Fourier transform  of the Meyer wavelet function in the periodic setting
at fine resolution level.
\end{Description}
\begin{Usage}
\begin{verbatim}
fine_YM(j, j_max, deg)
\end{verbatim}
\end{Usage}
\begin{Arguments}
\begin{ldescription}
\item[\code{j}] Resolution level (positive integer) 
\item[\code{j\_max}] Maximum resolution level (positive integer)
\item[\code{deg}] Degree of Meyer wavelet (1,2,3) 
\end{ldescription}
\end{Arguments}
\begin{Value}
Fourier transform  of the Meyer wavelet function at resolution level $j$
\end{Value}
\begin{Author}\relax
Marc Raimondo
\end{Author}
\begin{References}\relax
Raimondo, M. and Stewart, M. (2007),
"The WaveD Transform in R", Journal of Statistical Software.
\end{References}
\begin{SeeAlso}\relax
\code{\LinkA{WaveD}{WaveD}},
\end{SeeAlso}
\begin{Examples}
\begin{ExampleCode}
fine_YM(9,10,3)
\end{ExampleCode}
\end{Examples}

\HeaderA{make.doppler}{Make Doppler signal}{make.doppler}
\keyword{internal}{make.doppler}
\begin{Description}\relax
Generate Doppler signal.
\end{Description}
\begin{Usage}
\begin{verbatim}
make.doppler(n)
\end{verbatim}
\end{Usage}
\begin{Arguments}
\begin{ldescription}
\item[\code{n}] sample size
\end{ldescription}
\end{Arguments}
\begin{Value}
a vector of size $n$
\end{Value}
\begin{Author}\relax
Marc Raimondo
\end{Author}
\begin{References}\relax
Raimondo, M. and Stewart, M. (2007),
"The WaveD Transform in R", Journal of Statistical Software.
\end{References}
\begin{SeeAlso}\relax
\code{\LinkA{waved.example}{waved.example}}
\end{SeeAlso}
\begin{Examples}
\begin{ExampleCode}
plot(seq(0,1,le=1000),make.doppler(1000),type='l')
\end{ExampleCode}
\end{Examples}

\HeaderA{make.lidar}{Make LIDAR signal}{make.lidar}
\keyword{internal}{make.lidar}
\begin{Description}\relax
Generate artificial LIDAR signal.
\end{Description}
\begin{Usage}
\begin{verbatim}
make.lidar(n)
\end{verbatim}
\end{Usage}
\begin{Arguments}
\begin{ldescription}
\item[\code{n}] sample size
\end{ldescription}
\end{Arguments}
\begin{Value}
a vector of size $n$
\end{Value}
\begin{Author}\relax
Marc Raimondo
\end{Author}
\begin{References}\relax
Raimondo, M. and Stewart, M. (2007),
"The WaveD Transform in R", Journal of Statistical Software.
\end{References}
\begin{SeeAlso}\relax
\code{\LinkA{waved.example}{waved.example}}
\end{SeeAlso}
\begin{Examples}
\begin{ExampleCode}
plot(seq(0,1,le=1000),make.lidar(1000),type='l')
\end{ExampleCode}
\end{Examples}

\HeaderA{maxithresh}{Returns maxiset threshold}{maxithresh}
\keyword{internal}{maxithresh}
\begin{Description}\relax
Returns maxiset threshold for specified resolution levels.
\end{Description}
\begin{Usage}
\begin{verbatim}
maxithresh(data, g, L = 2, F = (log2(length(data)) - 1), eta = sqrt(6))
\end{verbatim}
\end{Usage}
\begin{Arguments}
\begin{ldescription}
\item[\code{data}] ~~Describe \code{data} here~~ 
\item[\code{g}] ~~Describe \code{g} here~~ 
\item[\code{L}] ~~Describe \code{L} here~~ 
\item[\code{F}] ~~Describe \code{F} here~~ 
\item[\code{eta}] ~~Describe \code{eta} here~~ 
\end{ldescription}
\end{Arguments}
\begin{Details}\relax
~~ If necessary, more details than the description above ~~
\end{Details}
\begin{Value}
Returns maxiset threshold for coarse resolution level equal to L, 
and wavelet resolution levels L,...,F.
\end{Value}
\begin{Author}\relax
Marc Raimondo
\end{Author}
\begin{References}\relax
Johnstone, I., Kerkyacharian, G., Picard, D. and Raimondo, M.  (2004), 
`Wavelet deconvolution in a periodic
setting', {\em Journal of the Royal Statistical Society, Series B} {\bf
66}(3),~547--573.  with discussion pp.627-652.

Raimondo, M. and Stewart, M. (2006),
`The WaveD Transform in R', preprint, School and Mathematics and Statistics,
University of Sydney.
\end{References}
\begin{SeeAlso}\relax
\code{\LinkA{WaveD}{WaveD}}
\end{SeeAlso}
\begin{Examples}
\begin{ExampleCode}
library(waved)
data=waved.example(TRUE,FALSE)
maxithresh(data$lidar.noisy,data$g,L=3,F=7)
\end{ExampleCode}
\end{Examples}

\HeaderA{multires}{Create Multi-resolution Plot}{multires}
\keyword{internal}{multires}
\begin{Description}\relax
Depicts wavelet coefficients according to time and resolution level.
\end{Description}
\begin{Usage}
\begin{verbatim}
multires(wcUntrimmed, lowest = 3, coarse = 3, highestplot = NULL, descending = FALSE, sc = 1)
\end{verbatim}
\end{Usage}
\begin{Arguments}
\begin{ldescription}
\item[\code{wcUntrimmed}] Vector of wavelet coefficients; must be of dyadic length.
\item[\code{lowest}] Lowest resolution level.
\item[\code{coarse}] Coarse resolution level; same as lowest.
\item[\code{highestplot}] Highest resolution level.
\item[\code{descending}] logical indicating whether resolutions are depicted with highest at the top of the plot (FALSE, the default), or at the bottom (TRUE).
\item[\code{sc}] graphical scaling parameter of heights of lines representing wavelet coefficients; default is 1.
\end{ldescription}
\end{Arguments}
\begin{Value}
Depicts wavelet coefficients according to time (horizontal axis)
and resolution level lowest, lowest+1,...,highestplot.
\end{Value}
\begin{Author}\relax
Marc Raimondo and Michael Stewart
\end{Author}
\begin{References}\relax
Donoho, D. and Raimondo, M.  (2004),
`Translation invariant deconvolution in a periodic setting', {\em The
International Journal of Wavelets, Multiresolution and Information
Processing} {\bf 14}(1),~415--423.

Johnstone, I., Kerkyacharian, G., Picard, D. and Raimondo, M.  (2004), 
`Wavelet deconvolution in a periodic
setting', {\em Journal of the Royal Statistical Society, Series B} {\bf
66}(3),~547--573.  with discussion pp.627-652.

Raimondo, M. and Stewart, M. (2006),
`The WaveD Transform in R', preprint, School and Mathematics and Statistics,
University of Sydney.
\end{References}
\begin{SeeAlso}\relax
\code{\LinkA{WaveD}{WaveD}}
\end{SeeAlso}
\begin{Examples}
\begin{ExampleCode}
library(waved)
data=waved.example(TRUE,FALSE)
lidar.w=FWaveD(data$lidar.blur,data$g,F=7)
multires(lidar.w,lo=3,hi=7)
\end{ExampleCode}
\end{Examples}

\HeaderA{phyHAT}{Meyer scaling function (Fourier domain).}{phyHAT}
\keyword{internal}{phyHAT}
\begin{Description}\relax
Compute Meyer Scaling function in the Fourier domain.
\end{Description}
\begin{Usage}
\begin{verbatim}
phyHAT(x, deg)
\end{verbatim}
\end{Usage}
\begin{Arguments}
\begin{ldescription}
\item[\code{x}] Abscissa (frequency) values for  evaluation.
\item[\code{deg}] The degree of the Meyer wavelet, either 1, 2, or 3 
\end{ldescription}
\end{Arguments}
\begin{Value}
Meyer scaling function at frequency $x$.
\end{Value}
\begin{Author}\relax
Marc Raimondo
\end{Author}
\begin{References}\relax
Raimondo, M. and Stewart, M. (2007),
"The WaveD Transform in R", Journal of Statistical Software.
\end{References}
\begin{SeeAlso}\relax
\code{\LinkA{WaveD}{WaveD}}
\end{SeeAlso}
\begin{Examples}
\begin{ExampleCode}
plot(seq(-2,2,le=1000),abs(phyHAT(seq(-2,2,le=1000),3)),type='l')

\end{ExampleCode}
\end{Examples}

\HeaderA{plot.wvd}{Plot  wvd objects}{plot.wvd}
\keyword{internal}{plot.wvd}
\begin{Description}\relax
Plot  wvd objects
\end{Description}
\begin{Usage}
\begin{verbatim}
plot(x,...) 
\end{verbatim}
\end{Usage}
\begin{Arguments}
\begin{ldescription}
\item[\code{x}] A list created by the WaveD function 
\end{ldescription}
\end{Arguments}
\begin{Value}
Graphical output only.
\end{Value}
\begin{Author}\relax
Marc Raimondo
\end{Author}
\begin{References}\relax
Raimondo, M. and Stewart, M. (2007),
"The WaveD Transform in R", Journal of Statistical Software.
\end{References}
\begin{SeeAlso}\relax
\code{\LinkA{WaveD}{WaveD}}
\end{SeeAlso}
\begin{Examples}
\begin{ExampleCode}
library(waved)
data=waved.example(TRUE,FALSE)
doppler.wvd=WaveD(data$doppler.noisy,data$g)
plot(doppler.wvd)

\end{ExampleCode}
\end{Examples}

\HeaderA{plotspec}{Plot spectrum}{plotspec}
\keyword{internal}{plotspec}
\begin{Description}\relax
This function plots 
the log spectrum and the optimal noise threshold
which is used in WaveD fit.
\end{Description}
\begin{Usage}
\begin{verbatim}
plotspec(g, s)
\end{verbatim}
\end{Usage}
\begin{Arguments}
\begin{ldescription}
\item[\code{g}] Sample of g or g + (Gaussian noise).
\item[\code{s}] Noise standard deviation.
\end{ldescription}
\end{Arguments}
\begin{Value}
Graphical output only.
\end{Value}
\begin{Author}\relax
Marc Raimondo
\end{Author}
\begin{References}\relax
Raimondo, M. and Stewart, M. (2007),
"The WaveD Transform in R", Journal of Statistical Software.
\end{References}
\begin{SeeAlso}\relax
\code{\LinkA{WaveD}{WaveD}}
\end{SeeAlso}
\begin{Examples}
\begin{ExampleCode}plotspec(sin(2*pi*seq(0,1,le=1024)),0.01) \end{ExampleCode}
\end{Examples}

\HeaderA{projFj}{Projection onto $F_j$}{projFj}
\keyword{internal}{projFj}
\begin{Description}\relax
Compute the projection of $f$ onto $F_j$ (fine resolution)).
\end{Description}
\begin{Usage}
\begin{verbatim}
projFj(beta, n, deg)
\end{verbatim}
\end{Usage}
\begin{Arguments}
\begin{ldescription}
\item[\code{beta}] vector of wavelet coefficients of $f$
\item[\code{n}] sample size
\item[\code{deg}] The degree of the Meyer wavelet, either 1, 2, or 3.
\end{ldescription}
\end{Arguments}
\begin{Value}
the projection of $f$ onto $F_j$
\end{Value}
\begin{Author}\relax
Marc Raimondo
\end{Author}
\begin{References}\relax
Raimondo, M. and Stewart, M. (2007),
"The WaveD Transform in R", Journal of Statistical Software.
\end{References}
\begin{SeeAlso}\relax
\code{\LinkA{WaveD}{WaveD}}
\end{SeeAlso}
\begin{Examples}
\begin{ExampleCode}
 plot(projFj(rnorm(1024),1024,3))
\end{ExampleCode}
\end{Examples}

\HeaderA{projVj}{Projection onto $V_j$}{projVj}
\keyword{internal}{projVj}
\begin{Description}\relax
Compute the projection of $f$ onto $V_j$.
\end{Description}
\begin{Usage}
\begin{verbatim}
projVj(beta, n, deg)
\end{verbatim}
\end{Usage}
\begin{Arguments}
\begin{ldescription}
\item[\code{beta}] vector of wavelet coefficients of $f$
\item[\code{n}] sample size
\item[\code{deg}] The degree of the Meyer wavelet, either 1, 2, or 3.
\end{ldescription}
\end{Arguments}
\begin{Value}
the projection of $f$ onto $V_j$
\end{Value}
\begin{Author}\relax
Marc Raimondo
\end{Author}
\begin{References}\relax
Raimondo, M. and Stewart, M. (2007),
"The WaveD Transform in R", Journal of Statistical Software.
\end{References}
\begin{SeeAlso}\relax
\code{\LinkA{WaveD}{WaveD}}
\end{SeeAlso}
\begin{Examples}
\begin{ExampleCode}
 plot(projVj(rnorm(512),1024,3))
\end{ExampleCode}
\end{Examples}

\HeaderA{projWj}{Projection onto $W_j$}{projWj}
\keyword{internal}{projWj}
\begin{Description}\relax
Compute the projection of $f$ onto $W_j$ (details).
\end{Description}
\begin{Usage}
\begin{verbatim}
projWj(beta, n, deg)
\end{verbatim}
\end{Usage}
\begin{Arguments}
\begin{ldescription}
\item[\code{beta}] vector of wavelet coefficients of $f$
\item[\code{n}] sample size
\item[\code{deg}] The degree of the Meyer wavelet, either 1, 2, or 3.
\end{ldescription}
\end{Arguments}
\begin{Value}
the projection of $f$ onto $W_j$
\end{Value}
\begin{Author}\relax
Marc Raimondo
\end{Author}
\begin{References}\relax
Raimondo, M. and Stewart, M. (2007),
"The WaveD Transform in R", Journal of Statistical Software.
\end{References}
\begin{SeeAlso}\relax
\code{\LinkA{WaveD}{WaveD}}
\end{SeeAlso}
\begin{Examples}
\begin{ExampleCode}
 plot(projWj(rnorm(512),1024,3))
\end{ExampleCode}
\end{Examples}

\HeaderA{psyHAT}{Meyer wavelet function (Fourier domain).}{psyHAT}
\keyword{internal}{psyHAT}
\begin{Description}\relax
Compute Meyer wavelet function in the Fourier domain.
\end{Description}
\begin{Usage}
\begin{verbatim}
psyHAT(x, deg)
\end{verbatim}
\end{Usage}
\begin{Arguments}
\begin{ldescription}
\item[\code{x}] Abscissa (frequency) values for  evaluation.
\item[\code{deg}] The degree of the Meyer wavelet, either 1, 2, or 3 
\end{ldescription}
\end{Arguments}
\begin{Value}
Meyer scaling function at frequency $x$.
\end{Value}
\begin{Author}\relax
Marc Raimondo
\end{Author}
\begin{References}\relax
Raimondo, M. and Stewart, M. (2007),
"The WaveD Transform in R", Journal of Statistical Software.
\end{References}
\begin{SeeAlso}\relax
\code{\LinkA{WaveD}{WaveD}}
\end{SeeAlso}
\begin{Examples}
\begin{ExampleCode}
plot(seq(-2,2,le=1000),abs(psyHAT(seq(-2,2,le=1000),3)),type='l')

\end{ExampleCode}
\end{Examples}

\HeaderA{rot90}{Rotate matrix 90 degrees}{rot90}
\keyword{internal}{rot90}
\begin{Description}\relax
Rotate matrix 90 degrees
\end{Description}
\begin{Usage}
\begin{verbatim}
rot90(x)
\end{verbatim}
\end{Usage}
\begin{Arguments}
\begin{ldescription}
\item[\code{x}] A square matrix. 
\end{ldescription}
\end{Arguments}
\begin{Value}
90 degrees rotation of $x$.
\end{Value}
\begin{Author}\relax
Marc Raimondo
\end{Author}
\begin{References}\relax
Raimondo, M. and Stewart, M. (2007),
`The WaveD Transform in R', Journal of Statistical Software.
\end{References}
\begin{SeeAlso}\relax
\code{\LinkA{WaveD}{WaveD}}
\end{SeeAlso}
\begin{Examples}
\begin{ExampleCode} rot90(1:5)\end{ExampleCode}
\end{Examples}

\HeaderA{scale}{Estimates standard deviation of noise}{scale}
\keyword{internal}{scale}
\begin{Description}\relax
Estimates standard deviation of noise in the nonparametric 
signal+(Gaussian noise)
regression  model. Input vector must be of dyadic length
and assumes a regular grid.
\end{Description}
\begin{Usage}
\begin{verbatim}
scale(yobs, L=3, deg=3)
\end{verbatim}
\end{Usage}
\begin{Arguments}
\begin{ldescription}
\item[\code{yobs}] a vector of dyadic length representing signal+(Gaussian noise)
\item[\code{L}] lowest resolution level
\item[\code{deg}] degree of Meyer wavelet
\end{ldescription}
\end{Arguments}
\begin{Value}
Returns a positive 
estimate  of the standard deviation of noise in the nonparametric 
regression  model.
\end{Value}
\begin{Author}\relax
Marc Raimondo
\end{Author}
\begin{References}\relax
Raimondo, M. and Stewart, M. (2006),
`The WaveD Transform in R', preprint, School and Mathematics and Statistics,
University of Sydney.
\end{References}
\begin{SeeAlso}\relax
\code{\LinkA{WaveD}{WaveD}}
\end{SeeAlso}
\begin{Examples}
\begin{ExampleCode}
library(waved)
data=waved.example(TRUE,FALSE)
scale(data$lidar.noisy)
\end{ExampleCode}
\end{Examples}

\HeaderA{scaling\_YM}{Meyer scaling function, Fourier domain.}{scaling.Rul.YM}
\keyword{internal}{scaling\_YM}
\begin{Description}\relax
Generate Fourier transform  of the Meyer scaling function in the periodic setting
at fine resolution level.
\end{Description}
\begin{Usage}
\begin{verbatim}
scaling_YM(j, j_max, deg)
\end{verbatim}
\end{Usage}
\begin{Arguments}
\begin{ldescription}
\item[\code{j}] Resolution level (positive integer) 
\item[\code{j\_max}] Maximum resolution level (positive integer)
\item[\code{deg}] Degree of Meyer wavelet (1,2,3) 
\end{ldescription}
\end{Arguments}
\begin{Value}
Fourier transform  of the Meyer scaling function at resolution level $j$
\end{Value}
\begin{Author}\relax
Marc Raimondo
\end{Author}
\begin{References}\relax
Raimondo, M. and Stewart, M. (2007),
"The WaveD Transform in R", Journal of Statistical Software.
\end{References}
\begin{SeeAlso}\relax
\code{\LinkA{WaveD}{WaveD}},
\end{SeeAlso}
\begin{Examples}
\begin{ExampleCode}
scaling_YM(9,10,3)
\end{ExampleCode}
\end{Examples}

\HeaderA{speczoom}{Plot spectrum}{speczoom}
\keyword{internal}{speczoom}
\begin{Description}\relax
This function plots  the log spectrum of $g$.
\end{Description}
\begin{Usage}
\begin{verbatim}
speczoom(y_test, fenetre)
\end{verbatim}
\end{Usage}
\begin{Arguments}
\begin{ldescription}
\item[\code{y\_test}] Sample vector of $g$.  
\item[\code{fenetre}] Window (a positive integer). 
\end{ldescription}
\end{Arguments}
\begin{Value}
log-spectrum of $g$.
\end{Value}
\begin{Author}\relax
Marc Raimondo
\end{Author}
\begin{References}\relax
Raimondo, M. and Stewart, M. (2007),
"The WaveD Transform in R", Journal of Statistical Software.
\end{References}
\begin{SeeAlso}\relax
\code{\LinkA{WaveD}{WaveD}}
\end{SeeAlso}
\begin{Examples}
\begin{ExampleCode}speczoom(sin(2*pi*seq(0,1,le=1024)),200) \end{ExampleCode}
\end{Examples}

\HeaderA{stoptime}{Optimal stoping time}{stoptime}
\keyword{internal}{stoptime}
\begin{Description}\relax
Compute the stoping time in the Fourier domain using noisy egein values.
\end{Description}
\begin{Usage}
\begin{verbatim}
stoptime(g, sigma)
\end{verbatim}
\end{Usage}
\begin{Arguments}
\begin{ldescription}
\item[\code{g}] A sample of the convolution kernel $g$. 
\item[\code{sigma}] Noise standard deviation.
\end{ldescription}
\end{Arguments}
\begin{Value}
\begin{ldescription}
\item[\code{M}] estimate of optimal Fourier frequency.
\item[\code{j1}] estimate of optimal resolution level.
\end{ldescription}
\end{Value}
\begin{Author}\relax
Marc Raimondo
\end{Author}
\begin{References}\relax
Raimondo, M. and Stewart, M. (2007),
"The WaveD Transform in R", Journal of Statistical Software.
\end{References}
\begin{SeeAlso}\relax
\code{\LinkA{WaveD}{WaveD}}
\end{SeeAlso}
\begin{Examples}
\begin{ExampleCode}
stoptime(log(abs(sin(2*pi*seq(0,1,le=1024)))),1)
\end{ExampleCode}
\end{Examples}

\HeaderA{summary.wvd}{Summary of wvd objects}{summary.wvd}
\keyword{internal}{summary.wvd}
\begin{Description}\relax
Provide a summary of wvd objects
\end{Description}
\begin{Usage}
\begin{verbatim}
summary.wvd(object,...)
\end{verbatim}
\end{Usage}
\begin{Arguments}
\begin{ldescription}
\item[\code{object}] A list created by the WaveD function  
\end{ldescription}
\end{Arguments}
\begin{Value}
Text output only.
\end{Value}
\begin{Author}\relax
Marc Raimondo
\end{Author}
\begin{References}\relax
Raimondo, M. and Stewart, M. (2007),
"The WaveD Transform in R", Journal of Statistical Software.
\end{References}
\begin{SeeAlso}\relax
\code{\LinkA{WaveD}{WaveD}}
\end{SeeAlso}
\begin{Examples}
\begin{ExampleCode}
library(waved)
data=waved.example(TRUE,FALSE)
doppler.wvd=WaveD(data$doppler.noisy,data$g)
summary(doppler.wvd)
\end{ExampleCode}
\end{Examples}

\HeaderA{threshsum}{Show threshold effects}{threshsum}
\keyword{internal}{threshsum}
\begin{Description}\relax
Provide a summary of threshold effects in WaveD fit.
\end{Description}
\begin{Usage}
\begin{verbatim}
threshsum(w.res, L = 3, F = (log2(length(w.res)) - 1))
\end{verbatim}
\end{Usage}
\begin{Arguments}
\begin{ldescription}
\item[\code{w.res}] A vector of wavelet coefficients 
\item[\code{L}] Low resolution level 
\item[\code{F}] Fine resolution level
\end{ldescription}
\end{Arguments}
\begin{Value}
A vector of length F-L+1, 
with ONES and ZEROS. The ZEROS show that
no coefficient remains at the corresponding resolution level.
\end{Value}
\begin{Author}\relax
Marc Raimondo
\end{Author}
\begin{References}\relax
Raimondo, M. and Stewart, M. (2007),
"The WaveD Transform in R", Journal of Statistical Software.
\end{References}
\begin{SeeAlso}\relax
\code{\LinkA{WaveD}{WaveD}}
\end{SeeAlso}
\begin{Examples}
\begin{ExampleCode}
library(waved)
data=waved.example(TRUE,FALSE)
doppler.wvd=WaveD(data$doppler.noisy,data$g)
threshsum(doppler.wvd$w,3,8)

\end{ExampleCode}
\end{Examples}

\HeaderA{waved.example}{WaveD  examples}{waved.example}
\keyword{nonparametric}{waved.example}
\begin{Description}\relax
Generate data sets and figures
to illustrate the WaveD function.
\end{Description}
\begin{Usage}
\begin{verbatim}
waved.example(pr = TRUE, gr=TRUE)
\end{verbatim}
\end{Usage}
\begin{Arguments}
\begin{ldescription}
\item[\code{pr}] If pr=TRUE (default)
uses the same parameters as in the reference paper below. If pr=FALSE
user level parameter specifications. 
\item[\code{gr}] If gr=TRUE (default) text and graphical displays are provided.   
\end{ldescription}
\end{Arguments}
\begin{Value}
\begin{ldescription}
\item[\code{lidar.noisy}] Noisy blurred LIDAR signal (Gaussian noise)
\item[\code{lidar.noisyT}] Noisy blurred LIDAR signal (Student $t_2$ noise)
\item[\code{doppler.noisy}] Noisy blurred Doppler signal (Gaussian noise)
\item[\code{doppler.noisyT}] Noisy blurred Doppler signal (Student $t_2$ noise)
\item[\code{lidar.blur}] Blurred LIDAR signal
\item[\code{doppler.blur}] Blurred Doppler signal
\item[\code{t}] Rime vector scaled to  [0,1]
\item[\code{n}] Sample size
\item[\code{g}] Convolution kernel
\item[\code{lidar}] LIDAR signal
\item[\code{doppler}] Doppler signal.
\item[\code{seed}] Used in set.seed
\item[\code{sigma}] Noise standard deviation.
\item[\code{g.noisy}] Convolution kernel plus Gaussian noise.
\item[\code{g.noisyT}] Convolution kernel plus Student $t_2$ noise.
\item[\code{dip}] Degree of Ill-posedness.
\item[\code{k.scale}] Scale of the convolution kernel
\end{ldescription}
\end{Value}
\begin{Author}\relax
Marc Raimondo
\end{Author}
\begin{References}\relax
Raimondo, M. and Stewart, M. (2007),
"The WaveD Transform in R", Journal of Statistical Software.
\end{References}
\begin{SeeAlso}\relax
\code{\LinkA{WaveD}{WaveD}}
\end{SeeAlso}
\begin{Examples}
\begin{ExampleCode} 
data=waved.example(TRUE,FALSE)
\end{ExampleCode}
\end{Examples}

\HeaderA{wavelet\_YM}{Meyer wavelet function, Fourier domain.}{wavelet.Rul.YM}
\keyword{internal}{wavelet\_YM}
\begin{Description}\relax
Generate Fourier transform  of the Meyer scaling function in the periodic setting
at fine resolution level.
\end{Description}
\begin{Usage}
\begin{verbatim}
wavelet_YM(j, j_max, deg)
\end{verbatim}
\end{Usage}
\begin{Arguments}
\begin{ldescription}
\item[\code{j}] Resolution level (positive integer) 
\item[\code{j\_max}] Maximum resolution level (positive integer)
\item[\code{deg}] Degree of Meyer wavelet (1,2,3) 
\end{ldescription}
\end{Arguments}
\begin{Value}
Fourier transform  of the Meyer wavelet function at resolution level $j$
\end{Value}
\begin{Author}\relax
Marc Raimondo
\end{Author}
\begin{References}\relax
Raimondo, M. and Stewart, M. (2007),
"The WaveD Transform in R", Journal of Statistical Software.
\end{References}
\begin{SeeAlso}\relax
\code{\LinkA{WaveD}{WaveD}},
\end{SeeAlso}
\begin{Examples}
\begin{ExampleCode}
wavelet_YM(5,10,3)
\end{ExampleCode}
\end{Examples}

\end{document}
